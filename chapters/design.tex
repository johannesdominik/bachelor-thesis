\chapter{Design and Architecture}
\label{cha:DesignArchitecture}

This chapter builds on top of the concepts and use cases of the previous chapters and lays out the architecture and design of a typical decentralized application by using a more practical approach. After a short introduction into the used technologies there will be a detailed view on each important part of a modern decentralized web application. To keep this chapter more concise, only selected code snippets will be shown to describe certain aspects. However, the entire code base of the working application can be found on GitHub \cite{CherryPool}.

\section{Technology Stack}
\subsection{Solidity}
The main components of decentralized web applications are Smart Contracts. Smart Contracts can be written in a few different programming languages such as LLL, Serpent, Solidity, Vyper and Bamboo. Solidity is by far the most popular and currently the de-facto standard in writing Smart Contracts \cite{AntonopoulosWood2018}. Solidity is statically typed, has a JavaScript-like syntax and supports various popular programming concepts such as libraries and inheritance \cite{SolidityDocumentation}. Because there are no classes in Solidity, it is considered not as an object-oriented but as a contract-oriented language \cite{Solidity}.

\subsection{Truffle}
Truffle is a development environment which helps to develop decentralized applications on the Ethereum Virtual Machine \cite{Truffle}. It handles all the difficult parts starting from Smart Contract compilation, testing, linking, deployment and continuous integration. When developing more than just some simple Smart Contracts, Truffle is an important tool to save a lot of time.

\subsection{Ganache}
Because testing a decentralized application on the main Ethereum network would not be a good idea (and is expensive too), some sort of testing blockchain is needed. Ganache is a local blockchain for developers which provides everything to test a decentralized applications on the local machine \cite{Ganache}. It offers accounts with fake ether and works really well with Truffle, which is built by the same developers.

\subsection{Web3}
In order to link Smart Contracts to a graphical user interface such as a web client application, web3.js comes into play. It is a JavaScript library and provides all the necessary APIs in order to connect a blockchain backend to a web frontend \cite{Web3}. It also manages connections to specific wallet providers such as MetaMask.

\subsection{Mocha \& Chai}
Testing plays in blockchain development an even bigger role than in software development in general, because deployed Smart Contracts are immutable and bugs literally cost money. A flawed function with causes an endless loop whould burn the entire gas that was available for that function call without doing anything useful. Mocha is a JavaScript testing framework which makes it possible to test the functionality of each Smart Contract in a structured and automated way \cite{Mocha}. Together with the assertion library Chai \cite{Chai} and the integration into Truffle, Mocha releases its full potential as a fully fledged blockchain testing environment.

\subsection{React}
What would be the best Smart Contract without a graphical user interfaces which gives people the opportunity to interact with the blockchain in an easy and intuitive way. React is a JavaScript library for building user interfaces for the web and by far the most popular web framework out there \cite{React}. React is super fast because it uses a virtual DOM and provides the user a modern and intuitive experience.

\subsection{TypeScript}
While other frontend frameworks such as Angular come with TypeScript out of the box, you have to add additional dependencies in React yourself. When it comes to larger projects or whenever bugs could become expensive, it is a good idea to introduce some kind of type checking in JavaScript. TypeScript is a superset of JavaScript and helps to detect errors before they occur at runtime \cite{TypeScript}. In financial applications like this, there is usually an even higher focus on reliable, bug-free code.

\section{The SafeMath Library}
Arithmetic operations in Solidity wrap on overflow. This can lead to bugs, because most programmers assume, that it would raise an exception when an overflow occurs, as it is done in other high level programming languages. By introducing the library \texttt{SafeMath}, an entire class of bugs is being eliminated so it should be used everywhere possible.

\begin{program}
\caption{The SafeMath library.}
\label{prog:SafeMath}
\begin{GenericCode}
library SafeMath {
  function sub(uint256 _a, uint256 _b) internal pure returns (uint256) {
    assert(_b <= _a);
    return _a - _b;
  }

  function add(uint256 _a, uint256 _b) internal pure returns (uint256) {
    uint256 c = _a + _b;
    assert(c >= _a);
    return c;
  }
}
\end{GenericCode}
\end{program}

\section{The ERC20 Token Standard}
\label{sec:ERC20}
The ERC20 standard makes it possible to build standardized tokens on the Ethereum protocol which can be integrated into the existing ecosystem \cite{ERC20}. This is achieved through a common interface each token needs to implement. The standard was initially introduced in 2015 as an Ethereum Request for Comments (ERC). It received its name due to the automatic issue assignment on Github, which had the issue number 20 \cite{AntonopoulosWood2018} \cite{BadrHorrocksWu2018}.

\subsection{Example Implementation}
The function \texttt{totalSupply} returns the maximum amount available of this token in \textit{wei}, the smallest unit possible. The supply can be either static or dynamic. In this case, it is static assigned through the constructor when the smart Contract is created. The return type is a 256 bit unsigned integer which is the most common data type in Solidity.

\begin{GenericCode}
 uint256 private _totalSupply;
  
 function totalSupply() public view returns (uint256) {
   return _totalSupply;
 }	
\end{GenericCode}

Each token is responsible to keep track of all addresses holding that token. This can be stored using the data type \texttt{mapping} which is a key value pair of addresses and token amounts. The function \texttt{balanceOf} returns the token balance for a specific address.

\begin{GenericCode}
 mapping(address => uint256) private balances;
  
 function balanceOf(address _owner) public view returns (uint256 balance) {
   return balances[_owner];
 }
\end{GenericCode}

The functionality of sending tokens to a specific address is implemented in the \texttt{transfer} function. It takes an address and an amount as parameters and returns a boolean if the transaction was successful. The \texttt{require} function checks for a specific condition to be true. If this is not the case, the function will be reverted and all the gas goes back to the sender, which can be evaluated in Solidity using \texttt{msg.sender}. In order to prevent over- or underflow errors when performing additions and subtractions, a small library \texttt{SafeMath} \ref{prog:SafeMath} is being utilized. If everything worked well, the \texttt{Transfer} event will be emitted.

\begin{GenericCode}
function transfer(address _to, uint256 _value) public returns (bool success) {
  require(_value <= balances[msg.sender]);
  balances[msg.sender] = balances[msg.sender].sub(_value);
  balances[_to] = balances[_to].add(_value);
  emit Transfer(msg.sender, _to, _value);
  return true;
}
\end{GenericCode}

The function \texttt{transferFrom} is very similar to \texttt{transfer} with the only difference that the caller is not the one who sends the tokens. This is usually not a physical person but rather a Smart Contract of a decentralized application. Tokens can be sent only on behalf of someone else if the sender accepts the amount first. All funds that are already released for transactions are stored in another mapping specifying for each address which addresses are enabled to send what amount.

\begin{GenericCode}
mapping(address => mapping(address => uint256)) private allowed;

function transferFrom(address _from, address _to, uint256 _value) public returns (bool success) {
  require(_value <= balances[_from]);
  require(_value <= allowed[_from][msg.sender]);
  balances[_from] = balances[_from].sub(_value);
  allowed[_from][msg.sender] = allowed[_from][msg.sender].sub(_value);
  balances[_to] = balances[_to].add(_value);
  emit Transfer(_from, _to, _value);
  return true;
}
\end{GenericCode}

If the caller is not the sender of a transaction, funds need to be approved first. That is exactly what the function \texttt{approve} does. It authorizes the spender to spend a specific amount of the tokens of the address which called the function. Old funds that are already authorized will not be accumulated but overwritten. This makes it possible to revert an approval by calling the \texttt{approve} function another time with the value of zero. This function emits the \texttt{Approval} event.
\begin{GenericCode}
function approve(address _spender, uint256 _value) public returns (bool success) {
  allowed[msg.sender][_spender] = _value;
  emit Approval(msg.sender, _spender, _value);
  return true;
}
\end{GenericCode}

Because the data structure holding the allowed funds is usually private, a public interface is needed to find out how much remaining tokens can be spent on behalf of someone else. The function \texttt{allowance} takes two parameters and returns the amount of tokens that the spender is authorized to spend from the owner's address.
\begin{GenericCode}
function allowance(address _owner, address _spender) public view returns (uint256 remaining) {
  return allowed[_owner][_spender];
}
\end{GenericCode}

The events \texttt{Transfer} and \texttt{Approval} play an important role in the interface of an ERC20 token. The \texttt{Transfer} event is emitted every time when a transaction is about to be executed. The \texttt{Approval} event is being called every time some tokens need to be approved. Both events trigger some sort of action in the linked wallet application of the user. For example, the browser wallet MetaMask opens a popup window asking the user for confirmation, once an event is emitted. 
\begin{GenericCode}
event Transfer(address indexed _from, address indexed _to, uint256 _value);
event Approval(address indexed _owner, address indexed _spender, uint256 _value);
\end{GenericCode}

\section{Liquidity Pool Implementation}
There are a few different approaches on how to implement a lending functionality. One possibility would be the utilization of collateral like it is
done by Maker\cite{MakerDAO2021}. However, this section will describe how to implement a lending application using liquidity pools as it builds the
foundation for decentralized exchanges such as Uniswap\cite{Uniswap2020}.

As described in section \ref{subsec:liquiditypools}, liquidity pools always need to have a pair of digital assets.
As we are building on top of the Ethereum network, we are limited to Ether and Ethereum tokens. In this example, the liquidity pool holds Ether
and an ERC20 token called CherryToken, as implemented in section \ref{sec:ERC20}. Besides the token reference \texttt{cherryToken}, the smart contract
needs to keep track of the owner, rewards that are ready to be distributed to the liquidity providers and a mapping where the balances of the liquidity providers
are stored. In order to create a single measurement for the pool pair, a virtual liquidity token is being created. It can be calculated by multiplicating both assets
and computing the square root. Because dealing with floating point numbers is difficult in Solidity and this is a redundant information which can be calculated, it
is done by the client and not directly on the blochain. This is a common practice to store only absolute necessary data directly on the blockchain in order to save
memory space and gas fees.
\begin{GenericCode}
address private _owner;
CherryToken private _cherryToken;

uint256 private _availableRewards = 0;

mapping(address => uint256) private _liquidityTokenBalances;
\end{GenericCode}

On Ethereum, not only Externally Owned Accounts (EOAs) can receive Ether. Smart Contracts are able to receive payments by adding the \texttt{payable} modifier
to a function. Most contracts have an empty payable function, called fallback function, to prevent the loss of Ether which is being sent to the contract
without calling a specific function. The getter functions \texttt{getEthBalance}, \texttt{getCtnBalance}, \texttt{getAvailableRewards} and
\texttt{getLiquidityTokenBalance} provide the frontend with necessary information to implement a good user experience and, again, to keep as much
calculation off-chain.
\begin{GenericCode}
function () external payable {}

function getEthBalance() public view returns (uint256 balance) {
  return address(this).balance;
}

function getCtnBalance() public view returns (uint256 balance) {
  return _cherryToken.balanceOf(address(this));
}

function getAvailableRewards() public view returns (uint256 rewards) {
  return _availableRewards;
}

function getLiquidityTokenBalance(address owner) public view returns (uint256 funds) {
  return _liquidityTokenBalances[owner];
}
\end{GenericCode}

Adding liquidity works a little bit different for Ether and an ERC20 token, as Ether is a fundamental feature of Solidity and an ERC20 token is managed by
the smart contract itself. Like the fallback function, which just adds anonymized Ether to the liquidity pool, \texttt{addLiquidity} detects the address
of the sender in order to issue liquidity tokens. After a few integrity checks, the token liquidity is transferred to the contract and the balance of the liquidity
provider gets updated.
\begin{GenericCode}
function addLiquidity(uint256 ctnAmount, uint256 liquidityTokenAmount) public payable {
  require(msg.value > 0, "ETH amount cannot be 0");
  require(_cherryToken.balanceOf(msg.sender) >= ctnAmount, "not enough CTN funds");
  _cherryToken.transferFrom(msg.sender, address(this), ctnAmount);
  _liquidityTokenBalances[msg.sender] = _liquidityTokenBalances[msg.sender].add(liquidityTokenAmount);
}
\end{GenericCode}

Removing liquidity from the pool means withdrawing the lended funds and receiving an additional reward proportional to the pool share.
As the reward calculation is quite computational intensive and Soldity is not good at managing divisions and percentages, the reward will be computed
off-chain as well. Because this is not managed by the function \texttt{removeLiquidity} itself, there are quite a few \texttt{require} statements necessary,
to detect potential errors. The rewards are created using the \texttt{mint} function from the token and paid out proportional to the share of liquidity holdings
of the investor.
\begin{GenericCode}
function removeLiquidity(uint256 ethAmount, uint256 ctnAmount, uint256 liquidityTokenAmount, uint256 totalLiquidityTokens) public {
  require(ethAmount > 0, "eth amount cannot be 0");
  require(ctnAmount > 0, "ctn amount cannot be 0");
  require(getEthBalance() >= ethAmount, "not enough ETH in pool");
  require(getCtnBalance() >= ctnAmount, "not enough CTN in pool");
  require(_liquidityTokenBalances[msg.sender] > liquidityTokenAmount, "not enough liquidity tokens");
  uint256 ctnRewards = _availableRewards.mul(liquidityTokenAmount.div(totalLiquidityTokens));
  _availableRewards = _availableRewards.sub(ctnRewards);
  _cherryToken.mint(msg.sender, ctnRewards);
  _cherryToken.transfer(msg.sender, ctnAmount);
  msg.sender.transfer(ethAmount);
  _liquidityTokenBalances[msg.sender] = _liquidityTokenBalances[msg.sender].sub(liquidityTokenAmount);
}
\end{GenericCode}

\section{Exchange Implementation}
A truly decentralized exchange is dependent of its own liquidity pools. While the concept of a liquidity pool is quite complex, the actual implementation
of the exchange is a lot easier. It can be implemented in just two functions. One function for each exchange direction. Both of these functions follow
the same structure: Receiving funds, calculating fees, deducting fees and sending funds.

When exchanging from Ether to an ERC20 token, the function needs to be payable again. The incoming payment is accessible in the function by using the value
property of the message object. After checking if the payment is not empty, the function calculates the fees which will distributed to the liquidity providers.
This calculation can be either fixed or dynamically. On more sophisticated exchanges, the percentage of the fees uses an approximation of an exponential curve.
If the liquidity pool is very small, the fees are really high\footnote{up to 50\% and more}. As the pool gains more depth, the fees decrease into small fractions
of percentages but never will be zero.

All numeric asset representations are stored in wei, the smallest unit on the Ethereum Platform. One Ether is equivalent to one quintillion wei, which is a
number with 19 digits.

\begin{GenericCode}
function swapEthToCtn(uint256 ctnOutput, uint256 fees) public payable {
  require(msg.value > 0, "amount cannot be 0");
  _addFees(fees);
  require(getCtnBalance() >= ctnOutput, "not enough funds available");
  _cherryToken.transfer(msg.sender, ctnOutput);
}
\end{GenericCode}

After the fees are calculated, the function checks, if the smart contract has enough funds to pay the user back. If this is not the case, the transaction aborts
and the money will be transferred back to the user. If there are still enough funds available, the amount deducted by the fees will be paid back using the
\texttt{transfer} function from the ERC token standard.

The reversed function is not equipped with the \texttt{payable} modifier, as the incoming funds are an ERC20 token. Because the funds of the ERC20 tokens are
managed by the token itself, the exchange contract has no direct access to it. If a user sends token funds to the address of the exchange, the exchange has no
way to get notified about it. Therefore, the token transaction will take place in the contract function on behalf of the user. All the user has to do before
calling the exchange function, is to grant the exchange permission to spend the funds on his behalf using the \texttt{approve} function from the ERC20 token
standard.

Instead of just paying the desired amount to the exchange, the amount needs to be passed to the function as a parameter. As before, the amount is being checked
if it is empty to save the gas which would be wasted when executing a swap with zero funds. After that, the exchange rate and fees will be calculated. After
another safety check, the tokens will be finally sent from the user to the exchange and the exchange sends the Ether amount back to the user using the
built-in \texttt{transfer} function.

\begin{GenericCode}
  function swapCtnToEth(uint256 ctnInput, uint ethOutput, uint256 fees) public {
    require(ctnInput > 0, "amount cannot be 0");
    _addFees(fees);
    require(getEthBalance() >= ethOutput, "not enough funds available");
    _cherryToken.transferFrom(msg.sender, address(this), ctnInput);
    msg.sender.transfer(ethOutput);
  }
\end{GenericCode}

\section{Testing Smart Contracts}
Testing plays a crucial part in the development of decentralized applications. Once deployed, the smart contracts are immutable on the blockchain. Fixing bugs after the initial deployment can be expensive and tedious, as the affected smart contracts need to be re-deployed and might be incompatible with the previous version. When working with financial assets of other people, it is even more vital, to minimize the number of bugs. Therefore, most applications use one or more testing frameworks and use a three-phase deployment cycle, as described in more detail in section \ref{sec:deployment}.

As a developer, you have the choice to write tests natively in Solidity, or using an arbitrary testing framework based on JavaScript. When working with Truffle, the testing framework Mocha\cite{Mocha} and the assertion library Chai\cite{Chai} are preinstalled as the default. All tests that are in the testing subdirectory, will be executed using the command \texttt{truffle test}.

At the top of each test file, the artifacts of the used smart contracts, as well as Chai, need to be imported:
\begin{JsCode}
const CherryPool = artifacts.require('CherryPool');
const CherryToken = artifacts.require('CherryToken');

require('chai')
  .use(require('chai-as-promised'))
  .should();	
\end{JsCode}

The specific tests are to be implemented in a contract block, which takes two parameters: The name of the tested contract and an array of test accounts as parameters. As the first account is used to deploy the contract, it is labelled as the contract owner.

\begin{JsCode}
contract('CherryPool', ([owner, user]) => {
	// tests are here
}	
\end{JsCode}

While the contract compilation takes place before the tests are being executed, the smart contracts still need to be instantiated. This can be done using a \texttt{before} block, where all initialization tasks for the tests take place:

\begin{JsCode}
before(async () => {
  // load contracts
  cherryToken = await CherryToken.new();
  cherryPool = await CherryPool.new(cherryToken.address);
  await cherryToken.transfer(cherryPool.address, '1000000000000000000000');
  await web3.eth.sendTransaction({ from: owner, to: cherryPool.address, value: web3.utils.toWei('1') });
});	
\end{JsCode}

The actual tests are grouped together to tests that are similar to each other or that test the same functionality into \texttt{describe} and \texttt{it} blocks for the concrete unit tests. The estimated results of the test can be checkt with assertions or by appending \texttt{should.be.rejected} to a function call.

\begin{JsCode}
describe('CherryPool deployment', async () => {
  it('has no rewards yet', async () => {
    const rewards = await cherryPool.getAvailableRewards();
    assert.equal(web3.utils.fromWei(rewards), 0);
  });

  it('has initial Ether pool', async () => {
    const ethBalance = await cherryPool.getEthBalance();
    assert.equal(web3.utils.fromWei(ethBalance), 1);
  });

  it('has initial CherryToken pool', async () => {
    const ctnBalance = await cherryPool.getCtnBalance();
    assert.equal(web3.utils.fromWei(ctnBalance), 1000);
  });
});
\end{JsCode}

\section{Connecting Smart Contracts to Web Applications}
The connection between the web frontend and the smart contracts is done using the JavaScript framework Web3\cite{Web3}, which is maintained by the Ethereum core team. It contains a lot of useful utility functions as seen before such as \texttt{web3.eth.sendTransaction()} for making transaction or \texttt{web3.utils.fromWei()} to convert a number from wei into Ether.

When the user wants to connect to a wallet, Web3 needs to determine the network provider first. Once detected, a new instance of Web3 will be instantiated and is ready for use.

\begin{JsCode}
function loadWeb3() {
  if ((window as any).ethereum) {
    (window as any).web3 = new Web3((window as any).ethereum);
    (window as any).ethereum.enable().then();
    return true;
  } else if ((window as any).web3) {
    (window as any).web3 = new Web3((window as any).web3.currentProvider);
    return true;
  }

  window.alert('Non-Ethereum browser detected. You should consider trying MetaMask!');
  return false;
}	
\end{JsCode}

The accounts array can be accessed via \texttt{web3.eth.getAccounts()} and the network id is retrieved using \texttt{web3.net.getId()}. Each contract can be instantiated afterwards based on the compilation artifacts. In addition to the tests, the correct network needs to selected first.

\begin{JsCode}
const cherryPoolData = (CherryPool as any).networks[networkId];
if (cherryPoolData) {
  cherryPool = new web3.eth.Contract((CherryPool as any).abi, cherryPoolData.address);
} else {
  alert('CherryPool contract not deployed to detected network!');
  return null;
}	
\end{JsCode}

After these steps, the API is ready to be used by any UI component. Smart contract functions can be invoked using \texttt{call()} or \texttt{send()}. The function \texttt{call()} is used if the function does not alter the state of the blockchain. A typical use case for that is retreiving an account balance:

\begin{JsCode}
account.cherryToken.methods.balanceOf(account.address).call().then((ctnBalance: number) => {
  setCherryTokenBalance(web3.utils.fromWei(ctnBalance.toString()));
});
          
\end{JsCode}

If a feature needs to alter the state of the blockchain, the function \texttt{send()} needs to be used. As every state modification costs gas, this will usually lead to a confirmation dialog provided by the wallet of the user. Typical use cases for that are adding liquidity, exchanging assets or approving funds for other accounts:

\begin{JsCode}
account.cherryToken.methods.approve(
  account.cherryPool._address,
  web3.utils.toWei(ctnToSupply.toString())
).send({ from: account.address }).then();	
\end{JsCode}


\section{Smart Contract Deployment}
\label{sec:deployment}

A typical release cycle of a blockchain application consists of three testing environments. At first, the smart contracts will be deployed to a local blockchain such as Ganache\cite{Ganache}. That is the environment where most of the testing will take place. Once the application is quite stable, it will be migrated to a public test network. These networks provide more realistic conditions, as it behaves more like the main network. Assets on the test networks have no value, which is really difficult to implement in reality. In order to prevent a value increase, the test networks need to be redeployed from time to time.

The most important test networks on Ethereum are Ropsten, Kovan (using the Aura consensus protocol) and Rinkeby (using the Clique consensus protocol). These conensus protocols are specifically designed to prevent the creation of value, which is exactly the opposite what the conensus protocols Proof of Work and Proof of Stake are trying to achieve.
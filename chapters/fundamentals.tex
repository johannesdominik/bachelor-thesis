\chapter{Fundamentals}
\label{cha:Fundamentals}

\section{Cryptography}

The term \textit{cryptocurrency} already suggests, that blockchains rely heavily on cryptography. But cryptography is an even broader topic and exceeds by far the ranges of blockchains, cryptocurrencies and decentralization. Without crytography, our digital life would be very different today. Cryptography is the theory of data encryption, which means "secret writing" in Greek. Together with cryptoanalysis, which is the opposite (decoding data), cryptographic can be conflated as cryptology. Cryptography makes it possible to make data content unreadable for specific people, detect unauthorized data manipulation and guarantee the authenticity of a communication parter. What cryptography cannot achieve, is prevention of unauthorized data manipulation and the prevention that data is being read at all.

\subsection{Public Key Cryptography}
Public key cryptography are asymmetrical encryption methods that utilize both private and public keys. While symmetrical enryption methods such as the Data Encryption Standard (DES) and Advanced Encryption Standard (AES) use a single key to encrypt and decrypt a message, asymmetrical encryption methods use the public key of the recipient to decrypt the message and a private key to enrypt it. At the first glance it may seem that asymmetrical enryption methods are way better than symmetrical encryption methods and that they should be used all the time. But that is not the case. Asymmetrical encryption methods are less secure, more complex, more difficult to invent and slower than symmetrical encryption methods. Each type of encryption method has its own use cases so they actually complement each other. While symmetrical encryption methods are really good at encrypting data content, asymmetrical enryption methods are used to exchange the symmetrical keys and prove ownership and authenticity through digital signatures.

The most popular asymmetric enryption methods are RSA (Rivest, Shamii, Adelman), Diffie-Hellman and elliptic curve cryptography. RSA and Diffie-Hellman are the most common algorithms outside of the blockchain ecosystem and are used for example in banking, telecommunications and ecommerce. Because those are not used by blockchain technologies, we will focus on elliptic curve cryptography.
\subsection{Elliptic Curve Cryptography}
Elliptic Curve Cryptography is an asymmetric cryptographic algorithm, which is a lot more difficult to compute than RSA and Diffie-Hellman. Both Bitcoin and
Ethereum use the secp256k1 algorithm, which uses a 256 bit length. An elliptic curve is the visualization of a prime order, which is nothing else than a set of
points in a two dimensional coordinate system. An elliptic curve makes it possible to calculate a third point, by connecting two random points on the elliptic
curve with each other. This works everytime, except when the two points are completely vertical to each other. By multiplicating a randomly generated private key
with a specific point on the elliptic curve, the generator point, it is possible to create all points on the elliptic curve in an unforeseen way. The result is the
public key, consisting of a prefix, the x and the y coordinate.

\subsection{Hashing Functions}
A hashing function maps a string of arbitrary size to a value of fixed size. But not all hashing functions are suitable for cryptographic problems. 
A good hashing function aims to have as few collisions as possible. A collision occurs when two different input strings lead to the same output hash.
In order to prevent that as good as possible, each hash value should occur equally and a minimal modification of the input string should lead to a
completely new hash value. This is important because the more collisions occur, the more vulnerable is the hashing function for hackers. If a hacker
finds a string that generates the same hash value he can apply the real signature to his faked text \cite{Schmeh2007}.

\section{The Ethereum Blockchain}
\subsection{Blockchain Fundamentals}
Ethereum is an unbounded state machine, which consists of a globally, uniform state and a virtual machine that makes changes in that state. Blockchains like Ethereum
consist of several components. The foundation is a decentralized peer-to-peer network around the world, where everyone can participate. The Ethereum main network, is
after the main network of Bitcoin, the second largest peer-to-peer network. Altering the state is on a blockchain done through transactions, which play a key role in
every blockchain. Furthermore, a set of rules needs to be applied to the network in order to reach consensus. There are currently two major consenus algorithms,
Proof of Work and Proof of Stake, which will be discussed in section \ref{sec:storeofvalue}. The global state machine, which is in Ethereum's case the Ethereum
Virtual Machine (EVM) uses that conensus algorithm, to process the transactions according to the defined rules. After processing them, they are grouped together to
specific blocks and chained after each other in a public ledger. The compliance of those rules would not be possible without a game theoretical incentivization
scheme. Of course it is impossible to enforce those rules as strictly as in any kind of centralized system. But the technology can be designed in such a way, that
it becomes economically unfeasibly to harm the network in a critical way. For example, no one can prevent you from winning guaranteed the lottery. But the system
is designed in such a way that it would cost a lot more buying all lottery tickets in contrast to the anticipated win.

So far, these are all theoretical concepts. In order to become a real blockchain, there needs to be one or more open-source implementation of that protocol. The
network can be as decentralized as possible, if not everyone has the opportunity to access the source code, there is still a little bit of trust involved, which
should be minimalized in any type of decentralized system.

\subsection{Clients}
Clients are exactly those open-source implementations, which are the primary interface to the peer-to-peer network. On Ethereum, there are a lot of different
implementations written in many different programming languages. The most popular clients are Parity, written in Rust, and Geth which is written in Go. Each client
is able to perform certain tasks on the blockchain, depending on the needs of the user. Almost every client serves as a wallet consisting of addresses and private and
public keys in order to have access to specific digital assets. Clients, which keep track of the entire blockchain are called full nodes. Full nodes have certain
hardware conditions because the entire blockchain can consume a lot of memory space. They play an important role to secure the network and make it more decentralized.
Some other implmentations can participate on the consensus algorithm to evaluated the next block, using either mining or staking. While each of these node types
can be combined in an arbitrary order, almost every node consist at least of a basic routing functionality to connect to other nodes.

\subsection{Wallets}
Wallets are the main interfaces for an average user to interact with a blockchain. They come in a lot of different forms, varying of usability, security, platform,
architecture and more. A common approach is to categorize them by their platforms. There can be web applications such as MyEtherWallet, browser extensions such as
MetaMask, desktop, mobile or crossplatform applications. These types of wallets are all considered as hot wallets, since their are connected constantly to the
internet. While this is very convenient for the user, it comes with a certain security risk, as the private key could be stolen by other people. Cold wallets,
however, aim to keep the private key disconnected from the internet as good as possible. This could be done using small devices called hard wallets, where the private
key never leaves the device, or through paper wallets, where the user writes down his private key on a piece of paper and stores that in a secure place.

The management of the private keys can be done in two ways. The older and simpler approach is by storing the keys in a nondeterministic way, which means generating
each private key independenly from each other based on a cryptographic random number generator. Nondeterministic wallets have certain disadvantages and should not be
used anymore today, except for simple tests.

The more modern approach are deterministic wallets, where all private keys are derived from a seed through the use of a one-way hash function. They are easy to
backup, import and export. The seed is usually derived from mnemonic code words, which makes it possible to build up a hierarchical tree structure with child keys.
The different branches could be logically organized for company departments or account categories. The public keys can then be generated on an untrusted server
without having access to the private keys.

\subsection{Transactions}
Transactions are the heart of the blockchain as they are the actual data which is stored in a data block. It is important to understand that there are not real
units of a coin stored on the blockchain. There are just discrete outputs, stored as integer values, which give people who know the private key the right to spend
them. Transactions consist of certain inputs and certain outputs, transferring the ownership of those atomic units of value. When someone says: "A wallet received
Ether", he just means that there is a new transaction output that can be spent with the private key of that wallet. Therefore the balance of a wallet is the sum of
all transaction outputs for that wallet. After a transaction is signed with a private key it will be submitted into the mempool, where they wait to be included into
a block. The first transaction of a block is a special transaction, known as the coinbase transaction. It does not consume a transaction output but pays new minted
coins to the miners.

But those are not the only incentives for miners or stakers. Each transaction can also include some transaction fees which are very important
for the blockchain for several reasons. As mentioned before, fees compensate miners and serve as a security mechanism, making it economically infeasible for attackers
to flood the network with transactions. As the fees are collected by that miner who mines the block which records that transaction, he will prefer transactions with
higher fees. Therefore transactions with higher fees are always processed faster than transactions with none or low fees. Depending on the current market situation,
it might be possible that the fees are so low, that the transaction will never be included into a block. In order to prevent that, fee estimation services calculate
the appropriate fee, based on capacity and the fees offered by competing transactions. Fees are not stored directly in a transaction, as every additional storage
costs money. Fees are the implicit difference between all transaction inputs and outputs.

\section{Smart Contracts}
Smart Contracts are programs that are executed by the Ethereum Virtual Machine (EVM). There are serveral programming languages that are specifically designed to
program Smart Contracts such as Serpent, Vyper and Bamboo, but Solidity is by far the most common. Smart Contracts have some characteristics that are different from
other programs. First of all, they only run when they are called by a transaction. A transaction can by only initiated by Externally Owned Accounts (EOAs) but both
contract addresses and EOAs are able to accept transactions. Transactions can have either a value, data or both of them. When the data payload is set, it is very
likely that this transaction is addressed to call a function from a smart contract. In contrast to Bitcoin, Ethereum is turing-complete, so the programming languages
give the developer all kinds of freedom to do what they want like they are used to in other high level programming languages. While this sounds great in the first
place, it is very vulnerable to introduce errors in the code. When dealing with applications in the Decentralized Finance sector, this is the last thing you want.
It is not even possible to fix the errors because Smart Contracts are immutable. Once deployed to a blockchain, it stays there forever. The execution on the EVM
is deterministic, so every user, no matter on which machine, gets the same output. Because Smart Contracts are turing-complete it would be possible to write code
that never terminates, which costs resources of the network. In order to prevent that, the sender has to send gas fees, which pay for the execution of the Smart
Contract. Once, the contract is out of gas, the EVM stops executing the code. Because it is very hard to predict how long turing-complete programs will run, the sender
sets a gas limit on the transaction, stating the maximum amount of gas he is willing to pay. Any unspent gas is being transferred back to the sender.

\section{Decentralized Finance}
The financial system is cumbersome. Traditional finance in particular has still problems to adapt to the digital age. Financial institutions are slow, biogratic
and based on a lot of trust. New online companies try to solve that problem with more automated software instead of trusted people, but they are still confined
to the boundaries of the ecosystem. PayPal, for example, tried to create its own digital currency but the government United States of America prohibited that.
Decentralized Finance brings the typical use cases of traditional finance into the digital world in a decentralized way. The goal is that nobody has to trust each
other but still can do business with anyone. It should be completely without censorship, so it is open for everyone, no matter which country in the world they live
in. And it should be more robust in terms of availability and security as there is no central party that would risk an outage of the entire system.

\subsection{State of the Art}
The term and concept of Decentralized Finance is still very young, many people refer to it as an experimental form of finance. It started to arouse interest in 2019
by a few technic enthusiastic people and became really popular in the world of digital assets in the middle of 2020. As of January 2021, there are approximately
\$26.2 billion locked in Smart Contracts on the Ethereum network \cite{DeFiPulse}, which is by far the most popular blockchain for Decentralized Finance. Despite the
tremendously high market dominance of Ethereum, DeFI is not confined to a single blockchan. There are already new blockchains arising that are specifically designed
for use cases in the Decentralized Finance area like DeFiChain\cite{DeFiChain}.
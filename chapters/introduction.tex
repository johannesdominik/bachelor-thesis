\chapter{Introduction}
\label{cha:Introduction}

\section{Motivation}
It is Thursday, Januray 28th, 2021. The stock market is currently experiencing very tempestuous times. The online community \textit{WallStreetBets} (WSB)
on the social media platform \textit{Reddit} managed to create a tremendously strong hype across serveral securites such as \textit{GameStop}, rising
the price up to 1000\% and more in just a few days. All these securities were known as common stocks that are being sold short by big hedge fonds.
Due to the rapid price increase, a lot of big fonds had to liqudidate their short positions and lost serveral \$10 billion dollars. The american
broker \textit{Robinhood} is known to be a common choice as a trading platform for members of the WSB community. In order to prevent the big hedge
fonds from even more losses, Robinhood decided to manipulate the market by preventing its users from buying the affected stocks. However, selling those
stocks was still possible. This resulted in an enormous price drop and the company had to deal with a lot of criticism on social media platforms.

This is just one of many examples of how a centralized financial authority abused its power in favor of the enrichment of itself or of its allies.
It happens on a day to day basis and is a big flaw of the traditional financial system. Events like the \textit{GameStop short squeeze} led people to
the creation of a new financial ecosystem, which is not characterized by cencorship, geo blocking, single point of failure and intransparency. This is
possible by making use of cryptocurrencies in order to create a world wide decentralized financial network to allow people to be more independent and
regain control over their own finances. While the fundamentals of such a financial economy are completely different, the ways how to use financial
products did not change. Decentralized applications enable the possibility to implment those use cases on a blockchain-based network, opening entire
new opportunities for the financial industry.

\section{Goals}
The goal of this document is to provide a better understanding of the term \textit{decentralized finance}, its most popular use cases and
how those can be implemented on the Ethereum platform using the programming language \textit{Solidity} and a web frontend. As Ethereum is considered
as a general purpose blockchain, there is an infinite number of ideas which could be built on Ethereum. However, this document only focuses on the
biggest financial use cases on the example of a decentralized web application.

\section{Structure of the Thesis}
Chapter \ref{cha:Fundamentals} provides a brief introduction to the fundamentals of blockchain technology. The major use cases of the
financial industry are described in chapter \ref{cha:UseCasesDecentralizedFinance} and how they operate specifically on decentralized
platforms. The chapter \ref{cha:DesignArchitecture} explains how an application based on Ethereum could be implemented that features
the use cases from the previous chapter before giving a short overview on criticism, risks and a forecast in chapter
\ref{cha:ClosingRemarks}.